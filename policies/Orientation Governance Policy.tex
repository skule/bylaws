%Use XeLaTeX to compile.

\documentclass[12pt]{article}

%PACKAGES BEGIN
\usepackage{fancyhdr}
\usepackage{geometry}
	\geometry{left=1in, right=1in, top=1in, bottom=1in}
\usepackage{fontspec}
\setmainfont{Garamond}%Special font, use XeLaTeX to compile.
\usepackage{graphicx}
\usepackage{titlesec}
\usepackage[ampersand]{easylist}
\usepackage{hyperref}
% PACKAGES END

% FILE INFO
\author{University of Toronto Engineering Society}
\title{Digital Services Policy} %Bylaw number
\date{}
% FILE INFO

% TITLING FORMATTING BEGIN
\titleformat{\section}{\bfseries}{\ \thesection . }{12pt}{}
\ListProperties(Numbers=a, Numbers4=l, Numbers5=r, Start1*=\thesection, FinalMark={.}, Hang=true, Margin2=1cm, Margin3=2.5cm, Margin4=3.5cm, Margin5=4.5cm, Align=1cm, Align3=1.5cm, Hide4=3, Hide5=4)
\setcounter{section}{-1}
% TITLING FORMATTING END

% HEADER/FOOTER BEGIN
\pagestyle{fancy}
\fancyhf{}
\setlength{\headheight}{42pt}
\lhead{
	\includegraphics{../images/logo.png}
	}
\rhead{
	\textbf{University of Toronto Engineering Society} \\
	\textbf{Orientation Governance Policy} \\
	Last Revision: \today
	}
\rfoot{
	\thepage
	}
% HEADER/FOOTER END

\begin{document}

% TITLE PAGE BEGIN
\begin{titlepage}
\begin{center}
\topskip0pt
\vspace*{\fill}
\Large\bfseries\uppercase{
	Policy Number ``1234-56-78"
	
	Orientation Governance Policy
	
	University of Toronto Engineering Society
	}
\vspace*{\fill}
\end{center}
\vfill
\begin{flushright}
ADOPTED: January 20, 2018

LAST REVISED: \today
\end{flushright}
\end{titlepage}
\pagebreak
% TITLE PAGE END

% Setting numbering correctly.
\pagenumbering{arabic}
\setcounter{page}{1}

% Content begin
\section{General}
\begin{easylist}
	&& Purpose
		&&& To aid the Engineering Society in maintaining strong internal control over Orientation while protecting the institutional independence of Orientation and ensuring a quality Orientation experience for incoming first-year students.
	&& Overview
		&&& The Orientation Chair is the Project Director responsible for F!rosh Week as outlined in Bylaw 2, Section 1.16.
		&&& The Orientation Chair directly oversees a budget of approximately \$100,000, a group of 400 volunteers and the well-being of over 1000 incoming students. The Orientation Chair (or appointed delegates) represents the Engineering Society to the Faculty, other campus groups and divisions, and external stakeholders in the City of Toronto.
		&&& Given the substantial amount of discretionary authority the Orientation Chair possesses and the vast number of volunteers involved with Orientation, it is necessary for the Engineering Society to remain actively involved throughout the process.
		&&& Given the large number of individuals who hold positions in Orientation as well as other positions within the Engineering Society and the resulting conflicts of interest, it is necessary for Orientation to maintain institutional independence to protect the experience of incoming first-year students.
	&& Interpretation 
		&&& This policy operates notwithstanding Policy Number “2013-02-01”, the Policy on Complaints.
		&&& In any situations where this Policy contradicts the Policy on Complaints, this Policy shall have precedence.
		&&& This policy does not override the University of Toronto Policy on Sexual Violence and Sexual Harassment nor any applicable municipal, provincial, and federal legislation (in particular, the Ontario Child, Youth and Family Services Act). All individuals involved with Orientation must comply with their obligations under the University of Toronto Policy on Sexual
Violence and Sexual Harassment as well as any applicable municipal, provincial, and federal legislation.
\end{easylist}

\section{Appointed Members}
\begin{easylist}
	&& The Orientation Chair must consult the President before appointing members to the Orientation Committee (Vice-Chairs, Sub-Committee Chairs, and Head Leedurs).
	&& Any appointed member of the Orientation Committee (Vice-Chairs, Sub-Committee Chairs, and Head Leedurs) may be removed from their position by a majority vote of a committee comprised of the Orientation Chair, the Assistant Director, First Year Student Success \& Transition, and the President.
\end{easylist}

\section{Project Management}
\begin{easylist}
	&& The Orientation Chair must provide a tentative schedule for Orientation to the President by the beginning of April.
	&& The Orientation Chair must provide a written monthly update to the President at the beginning of every month in the Summer Months.
\end{easylist}

\section{Sexual Violence Prevention and Response}
\begin{easylist}
	&& The Orientation Chair and Vice-Chairs are required to receive sexual violence prevention training from the University’s Sexual Violence Prevention and Support Centre before the start of August.
	&& The Orientation Chair is required to create a Sexual Violence Prevention and Response plan before August 15th, and this plan will be presented to all members of the Engineering Society taking part in F!rosh Week and F!rosh Week-adjacent events.
\end{easylist}

\section{Restricted Participation in Orientation}
\begin{easylist}
	&& List of Individuals
		&&& The Orientation Chair will maintain a list of individuals who are not permitted to participate in or engage with Orientation or Orientation-related events and activities.
		&&& The removal of individuals from the list will be done at the discretion of the Orientation
Chair, who will consult the President before removing any individual.
		&&& Unless removed by the Orientation Chair, individuals will remain on the list indefinitely.
		&&& The list will be shared only with the President and Orientation Vice-Chairs, except as needed to enforce the requirement that individuals do not participate in or engage with Orientation or Orientation-related events and activities.
			&&&& Determining with whom to share the list (including with other Officers of the Engineering Society) shall be done at the sole discretion of the Orientation Chair.
	&& Complaints and Investigation
		&&& Complaints alleging that an individual’s behaviour falls under a criterion outlined in Section 4.3.1 should be submitted directly to the Orientation Chair.
			&&&& If a person submitting a complaint is unsatisfied with the resolution of their complaint, the person should submit their complaint directly to the President, who will discuss the matter with the Orientation Chair.
		&&& Determining if an individual’s behaviour falls under a criterion outlined in Section 4.2.1, including the procedure and timeline of any investigation as well as any engagement with external parties, shall be done at the sole discretion of the Orientation Chair.
		&&& If the complaint is regarding the Orientation Chair, it should be submitted to the President. In this case, determining if the Orientation Chair’s behaviour falls under a criterion outlined in Section 4.2.1 shall be done at the sole discretion of the President, including whether to invoke procedures outlined in the Policy on Complaints.
	&& Standard of Behaviour
		&&& Individuals will not be permitted to participate in or engage with Orientation or Orientation-related events if there exist credible allegations that the individual has:
			&&&& committed sexual violence or sexual harassment (as defined in the University of Toronto Policy on Sexual Violence and Sexual Harassment),
			&&&& committed other forms of harassment (as defined in the University of Toronto Statement on Prohibited Discrimination and Discriminatory Harassment) such as use of racial slurs,
			&&&& attempted to engage a member of the incoming first-year class romantically or sexually, before the end of the first-year member's Orientation,
			&&&& committed or attempted to commit other acts deemed inappropriate by the Orientation Chair and President, or
			&&&& failed to report violations of this standard of behaviour to an appropriate authority (whether to appropriate individuals within Orientation and the Engineering Society, to appropriate individuals within the University of Toronto, or appropriate law enforcement authorities).
		&&& Different violations of the standard of behaviour may result in consequences of different severity and duration.
		&&& The Orientation Chair is responsible for ensuring that all individuals involved with Orientation are aware of the standard of behaviour outlined in this section, as well as with their obligations under the University of Toronto Policy on Sexual Violence and Sexual Harassment and any applicable municipal, provincial, and federal legislation (in particular, the Ontario Child, Youth and Family Services Act).
\end{easylist}

\end{document}