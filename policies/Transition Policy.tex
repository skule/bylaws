%Use XeLaTeX to compile.

%%%%%
% SET THIS TO THE DATE THAT THE CURRENT VERSION
% IS PASSED/TO BE PASSED BY THE OFFICERS OR BOARD
%%%%%
\newcommand{\revdate}{May 17, 2023}

\documentclass[12pt]{article}

%PACKAGES BEGIN
\usepackage{fancyhdr}
\usepackage{geometry}
	\geometry{left=1in, right=1in, top=1in, bottom=1in}
\usepackage{fontspec}
\usepackage[lining,tabular]{ebgaramond}%Special font, use XeLaTeX to compile.
\usepackage{graphicx}
\usepackage{titlesec}
\usepackage[ampersand]{easylist}
\usepackage{hyperref}
% PACKAGES END

% FILE INFO
\author{University of Toronto Engineering Society}
\title{Transition Policy} %Bylaw number
\date{}
% FILE INFO

% TITLING FORMATTING BEGIN
\titleformat{\section}{\centering\bfseries\large\uppercase}{Chapter\ \thesection \ - }{0ex}{}
\newcommand{\sectionbreak}{\clearpage}
\ListProperties(Numbers=a, Numbers4=l, Numbers5=r, Style2=\bfseries, Start1*=\thesection, Start2=0, FinalMark={.}, Hang=true, Margin2=0cm, Margin3=1cm, Margin4=2.5cm, Margin5=3.5cm, Align=1cm, Align3=1.5cm, Space2=.5cm, Hide4=3, Hide5=4)
\setcounter{section}{-1}
% TITLING FORMATTING END

% HEADER/FOOTER BEGIN
\pagestyle{fancy}
\fancyhf{}
\setlength{\headheight}{42pt}
\lhead{
	\includegraphics{../images/logo.png}
	}
\rhead{
	\textbf{University of Toronto Engineering Society} \\
	\textbf{Transition Policy} \\
	Last Revision: \revdate
	}
\rfoot{
	\thepage
	}
% HEADER/FOOTER END

\begin{document}

% TITLE PAGE BEGIN
\begin{titlepage}
\begin{center}
\topskip0pt
\vspace*{\fill}
\Large\bfseries\uppercase{
	Policy Number ``2022-04-16"

	Transition Policy

	University of Toronto Engineering Society
	}
\vspace*{\fill}
\end{center}
\vfill
\begin{flushright}
ADOPTED: April 16, 2022

LAST REVISED: \revdate
\end{flushright}
\end{titlepage}
\pagebreak
% TITLE PAGE END

% Setting numbering correctly.
\pagenumbering{arabic}
\setcounter{page}{1}

% Content begin
\section{General}
\vspace{5mm} %Header heigh consistency.
\begin{easylist}
	&& Purpose
		&&& The purpose of this document is to set guidelines for transitioning new members of the Engineering Society Council, to ensure that they have the necessary training to be effective in their roles.
		&&& Though all members of the Engineering Society Council should make an effort to transition in their successors, the guidelines in this document should be considered requirements for Officers, Project Directors, the Speaker, and the Chief Returning Officer.
	&& Definitions
		&&& ``Transition” shall refer to the process of training one’s successor for their new role.
	&& Enforcement
		&&& The President of the Engineering Society is responsible for transitioning their own successor and ensuring that the other outgoing Officers and Speaker are working to transition their successors.
		&&& Outgoing Officers are responsible for ensuring that the Project Directors they oversee have written their Transition Reports prior to the April Board of Directors Meeting.
		&&& The outgoing Speaker is responsible for ensuring that the Ombudsperson and Chief Returning Officer have written their Transition Reports prior to the April Board of Directors Meeting.
\end{easylist}

\section{Transition Reports}
\begin{easylist}
\ListProperties(Start2=0)
&& General
	&&& A Transition Report is a document which outlines the necessary information for the transition.
		&&&& A transition report shall be prepared by Officers, Project Directors, the Speaker, and the Chief Returning Officer.
		&&&& A \href{https://docs.google.com/document/d/1uN2ui3WaedRhSUtC0AQIJCpq0djgoHSq/edit?usp=sharing&ouid=111631590450340878953&rtpof=true&sd=true}{recommended template} for the transition report shall be available on skule.ca
	&&& The Transition Report should be prepared prior to the successor entering their new position and given to them as a digital document no more than one week following the Board of Directors meeting at which they formally enter the position.
\end{easylist}

\section{Transition Meetings}
\begin{easylist}
\ListProperties(Start2=0)
&& General
	&&& Transition Meetings shall be held between the outgoing and incoming Council member and shall consist of one or more meetings of appropriate length at the discretion of the outgoing Council member, but should consist of at least one (1) session of an hour length. The content of these meetings shall include but not be limited to:
		&&&& Practical training such as familiarizing the incoming Council member with necessary tools for their role.
		&&&& An opportunity for the incoming Council member to ask questions about their responsibilities.
	&&& A Transition Meeting should be completed prior to an incoming Officer taking on their role.
		&&&& Between Transition Meetings prior to formally taking on their role, incoming Officers are encouraged to familiarize themselves with the responsibilities of their new role.
	&&& A Transition Meeting should be scheduled within three (3) weeks following the Board of Directors meeting at which a Project Director, Speaker, or Chief Returning Officer is elected.

\end{easylist}

\end{document}
