%Use XeLaTeX to compile.

%%%%%
% SET THIS TO THE DATE THAT THE CURRENT VERSION
% IS PASSED/TO BE PASSED BY THE OFFICERS OR BOARD
%%%%%
\newcommand{\revdate}{March 29, 2023}

\documentclass[12pt]{article}

%PACKAGES BEGIN
\usepackage{fancyhdr}
\usepackage{geometry}
	\geometry{left=1in, right=1in, top=1in, bottom=1in}
\usepackage{fontspec}
\usepackage[lining,tabular]{ebgaramond}%Special font, use XeLaTeX to compile.
\usepackage{graphicx}
\usepackage{titlesec}
\usepackage[ampersand]{easylist}
\usepackage{hyperref}
% PACKAGES END

% FILE INFO
\author{University of Toronto Engineering Society}
\title{Policy on Complaints} %Bylaw number
\date{}
% FILE INFO

% TITLING FORMATTING BEGIN
\titleformat{\section}{\bfseries}{\ \thesection . }{12pt}{}
\ListProperties(Numbers=a, Numbers4=l, Numbers5=r, Start1*=\thesection, FinalMark={.}, Hang=true, Margin2=1cm, Margin3=2.5cm, Margin4=3.5cm, Margin5=4.5cm, Align=1cm, Align3=1.5cm, Hide4=3, Hide5=4)
\setcounter{section}{-1}
% TITLING FORMATTING END

% HEADER/FOOTER BEGIN
\pagestyle{fancy}
\fancyhf{}
\setlength{\headheight}{42pt}
\lhead{
	\includegraphics{../images/logo.png}
	}
\rhead{
	\textbf{University of Toronto Engineering Society} \\
	\textbf{Policy on Complaints} \\
	Last Revision: \revdate
	}
\rfoot{
	\thepage
	}
% HEADER/FOOTER END

\begin{document}

% TITLE PAGE BEGIN
\begin{titlepage}
\begin{center}
\topskip0pt
\vspace*{\fill}
\Large\bfseries\uppercase{
	Policy Number ``2013-02-01"

	Policy on Complaints

	University of Toronto Engineering Society
	}
\vspace*{\fill}
\end{center}
\vfill
\begin{flushright}
ADOPTED: October 26, 2013

LAST REVISED: \revdate
\end{flushright}
\end{titlepage}
\pagebreak
% TITLE PAGE END

% Setting numbering correctly.
\pagenumbering{arabic}
\setcounter{page}{1}

% Content begin
\section{General}
\begin{easylist}
	&& Purpose
		&&& To provide Engineering Society members with a formal procedure intended to ensure that complaints are handled fairly and consistently.
	&& Overview
		&&& It is a requirement under the University of Toronto Policy for Compulsory Non-Academic Incidental Fees that organizations receiving such fees have and adhere to an internal process for addressing complaints. Further, the existence and continual refinement of such a process is an organizational best practice that is in the interest of facilitating member participation in the Society
	&& This policy applies to any member that holds a position within the Engineering Society, including but not limited to, an Officer, Project Director, employee, member of the Board of Directors, or member of a Project Director’s team.
		&&& This policy applies to any of the persons listed in Section 0.3 for the entirety of their term, from when they take office of their position to when they are relieved of their position. This includes actions taken outside of their official capacity within their role.
	&& Complaints to which this policy applies include the misconduct of any member that holds a position within the Engineering Society as defined in Section 0.3. This policy defines an act of misconduct as any of the following, as interpreted by the investigator of the case:
		&&& any offense outlined in the University of Toronto Code of Student Conduct
		&&& any form of sexual violence or sexual harassment in the University of Toronto Policy on Sexual Violence and Sexual Harassment
		&&& any form of harassment outlined in the University of Toronto Statement on Prohibited Discrimination and Discriminatory Harassment
		&&& any violation of the Standard of Behavior defined in the Orientation Governance Policy, for complaints against members of the Orientation Committee
		&&& any form of defamation or slander that injures the complainant's reputation with false statements of fact
		&&& failure to perform duties as required by the position held by the subject of the complaint, or as directed by an overseeing body with the authority to so direct them
	&& All materials related to complaints, including without limitation the original complaint, any recommendation report, and communications with relevant parties, are strictly confidential to the extent practicable.
		&&& If a recommendation report is presented to the Board of Directors, it must be presented \textit{in camera}. However, if the Board chooses to execute the recommendation, the recommendation can no longer be confidential.
		&&& If a recommendation report is presented to a General Meeting, the version presented can no longer be confidential.
		&&& Multiple versions of a recommendation report may/should be produced, with different information redacted depending on the audience of each version in accordance with section 4.3.2.b.
\end{easylist}

\section{Submission of Complaints}
\begin{easylist}
	&& All complaints should be submitted to the Ombudsperson, except a complaint against the Ombudsperson, which should be directed to the Speaker.
		&&& If any person holding any office of the Society receives a formal complaint, they must refer the complainant to the Ombudsperson or
Speaker, as outlined in Section 1.1 respectively, to pursue an investigation.
	&& In the absence or unavailability of the Ombudsperson, the Speaker shall carry out all of their functions.
	&& Complaints may be submitted in person or by mail to the Society’s registered address, or by email to the appropriate person (as outlined in Section 1.1).
	&& Complaints may be submitted anonymously, so long as the identity of the complainant is not germane to the complaint itself.
\end{easylist}

\section{Investigator and Completeness of Complaints}
\begin{easylist}
	&& Complaints sent to the Ombudsperson and Speaker will be sent to the relevant personnel immediately for investigation as follows:
		&&& If the complaint concerns the conduct of one or more of the Directors of the Board (excluding the Speaker), the Officers, or the Ombudsperson, the Speaker will be the investigator.
		&&& If the complaint concerns the conduct of one or more of the Orientation Chairs, the President will be the investigator.
		&&& If the complaint concerns the conduct of one or more other members of the Orientation Committee, the Orientation Chair(s) will be the investigator.
		&&& If the complaint concerns the conduct of any other person(s), the Ombudsperson will investigate the complaint as the investigator.
	&& The investigator shall immediately acknowledge receipt of the complaint to the complainant, unless there is no means of doing so.
	&& Within 7 days, the investigator shall advise the complainant on whether the complaint submitted is complete.
		&&& A complaint will be considered complete if it:
			&&&& Concerns a person to which this Policy applies, as described in section 0.3; and
			&&&& Contains sufficient detail, such as an approximate date or timeline of the alleged conduct, for a reasonable person, with full access to the Society's documents, to ascertain the factual accuracy of the allegation.
		&&& If the complaint is deemed incomplete, the complaint shall not be investigated. To pursue the complaint, the complainant must to submit a new, complete complaint. This shall be communicated to the complainant, unless there is no means of doing so. The investigator shall archive the complaint.
\end{easylist}

\section{Investigation Process}
\begin{easylist}
	&& Upon advising that a complaint is complete, the investigator shall immediately take steps to ascertain whether the complaint actionably concerns misconduct under section 0.4 of this Policy committed by persons listed in section 0.3.
		&&& The investigator must contact all possible relevant parties in the complaint and offer to remain in contact throughout the investigation.
		&&& The investigator shall make the greatest feasible effort to determine all the facts of the case.
		&&& The investigator shall make a reasonable effort to determine what the appropriate conduct would have been and whether there have been previous similar cases.
		&&& All persons holding any office of the Society shall cooperate in the investigation of complaints to the fullest extent that is reasonably possible.
	&& At any point during the investigation of a complaint, the complainant can choose to withdraw their complaint by notifying the investigator, at which point the associated investigation shall cease immediately.
	&& The investigator must respond to the complainant, within 21 days of acknowledging the complaint, with an outcome or a notification for a defined time extension to the investigation period.
\end{easylist}

\section{Response and Outcomes to Complaints}
\begin{easylist}
	&& The investigator will respond to the complainant in writing to inform them of the close of the investigation. The response shall note whether the investigator found the complaint actionable.
	&& If the investigator finds the complaint inactionable, the investigation ends there. The possible reasons for inactionability are:
		&&& None of the conduct of the subject(s) of the complaint constitutes misconduct under section 0.4 of this Policy.
		&&& The misconduct was not committed during the time period in which the subject(s) held their position(s), except when the misconduct, had it been known, would likely have led to the subject(s) not being selected for their position(s).
		&&& The facts of the case do not prove, on the balance of the probabilities, that the misconduct alleged in the complaint was committed by the subject(s) of the complaint.
	&& If the investigator finds the complaint actionable, they must produce a recommendation report.
		&&& The recommendation report shall include:
			&&&& all the facts of the case found by the investigator
			&&&& the conduct the subject(s) of the complaint should have exhibited, if applicable and a matter of fact rather than opinion
			&&&& a history of any similar cases in the past, if extant
			&&&& reasoning for the investigator's recommendation(s), based on all of the above
			&&&& one or more recommendations
		&&& The recommendation report shall not include:
			&&&& rumors, speculation, nor other statements unsubstantiated by facts
			&&&& information likely to compromise the well-being of its source, when shared with parties other than the investigator
			&&&& personally identifying information, when shared with parties other than the investigator
		&&& The possible recommendations mentioned in section 4.3.1.e are to:
			&&&& Recall subject of complaint from their position.
			&&&& Mandate additional training.
			&&&& Require subject of complaint to issue official apology.
			&&&& Escalate matter to campus police.
			&&&& Escalate matter to governmental law enforcement.
			&&&& Escalate matter to the Office of the Faculty Dean.
			&&&& Escalate matter to the Office of the University Provost.
			&&&& Amend EngSoc Bylaws and/or Policies.
			&&&& Take some other specific action, with additional justification for such an extraordinary recommendation.
		&&& The recommendation report shall be confidentially (in accordance with section 0.5) provided to the following parties:
			&&&& the complainant (if possible)
			&&&& the subject(s) of the complaint
			&&&& the overseeing body
			&&&& the first body in the chain of oversight with the power to execute the recommendations
				&&&&& This body shall consider the recommendation(s) and decide whether to execute it/them fully, partially, or not at all.
				&&&&& For Project Directors, this body is the Board of Directors.
				&&&&& For Directors of the Board and Officers, this body is a General Meeting, unless the Board dispenses of this requirement by failing to pass the motion calling such a General Meeting.
	&& All complaints and related documents will be archived by the investigator, for the possibility of reopening a withdrawn case or for the use of reference as a precedent.
\end{easylist}

\section{Appeals Process}
\begin{easylist}
	&& As with any decision made in the Society, decisions on complaints may be appealed to the next higher overseeing body with the power to overturn the decision. For example:
		&&& Decisions by the Ombudsperson on the completeness or actionability of a complaint may be appealed to the Speaker.
		&&& Decisions by the Board of Directors on the disposition of the recommendation(s) related to a complaint may be appealed to a General Meeting.
		&&& Decisions at a General Meeting may not be appealed.
\end{easylist}

\end{document}
