%Use XeLaTeX to compile.

%%%%%
% SET THIS TO THE DATE THAT THE CURRENT VERSION
% IS PASSED/TO BE PASSED BY THE OFFICERS OR BOARD
%%%%%
\newcommand{\revdate}{April 14, 2023}

\documentclass[12pt]{article}

%PACKAGES BEGIN
\usepackage{fancyhdr}
\usepackage{geometry}
	\geometry{left=1in, right=1in, top=1in, bottom=1in}
\usepackage{fontspec}
\usepackage[lining,tabular]{ebgaramond}%Special font, use XeLaTeX to compile.
\usepackage{graphicx}
\usepackage{titlesec}
\usepackage[ampersand]{easylist}
\usepackage{hyperref}
% PACKAGES END

% FILE INFO
\author{University of Toronto Engineering Society}
\title{Policy on Affiliated Clubs} %Bylaw number
\date{}
% FILE INFO

% TITLING FORMATTING BEGIN
\titleformat{\section}{\centering\bfseries\large\uppercase}{Chapter\ \thesection \ - }{0ex}{}
\newcommand{\sectionbreak}{\clearpage}
\ListProperties(Numbers=a, Numbers4=l, Numbers5=r, Style2=\bfseries, Start1*=\thesection, Start2=0, FinalMark={.}, Hang=true, Margin2=0cm, Margin3=1cm, Margin4=2.5cm, Margin5=3.5cm, Align=1cm, Align3=1.5cm, Space2=.5cm, Hide4=3, Hide5=4)
\setcounter{section}{-1}
% TITLING FORMATTING END

% HEADER/FOOTER BEGIN
\pagestyle{fancy}
\fancyhf{}
\setlength{\headheight}{42pt}
\lhead{
	\includegraphics{../images/logo.png}
	}
\rhead{
	\textbf{University of Toronto Engineering Society} \\
	\textbf{Policy on Affiliated Clubs} \\
	Last Revision: \revdate
	}
\rfoot{
	\thepage
	}
% HEADER/FOOTER END

\begin{document}

% TITLE PAGE BEGIN
\begin{titlepage}
\begin{center}
\topskip0pt
\vspace*{\fill}
\Large\bfseries\uppercase{
	Policy Number ``2015-06-21"

	Policy on Affiliated Clubs

	University of Toronto Engineering Society
	}
\vspace*{\fill}
\end{center}
\vfill
\begin{flushright}
ADOPTED: December 12, 2015

LAST REVISED: \revdate
\end{flushright}
\end{titlepage}
\pagebreak
% TITLE PAGE END

% Setting numbering correctly.
\pagenumbering{arabic}
\setcounter{page}{1}

% Content begin
\section{General}
\vspace{5mm} %Header heigh consistency.
\begin{easylist}
	&& Purpose: To describe the procedure by which undergraduate student groups can become affiliated with the Engineering Society and outline the standards to be met by affiliated clubs.
	&& Definitions
		&&& The ``Engineering Society” shall refer to the University of Toronto Engineering Society, hereafter referred to as ``the Society.”
		&&& ``Members” shall mean members of the Society, as defined in Chapter 1 of Bylaw 1 - The Constitution.
		&&& The ``Affiliation Year” shall be defined from September 1st of any year to August 31st of the following year.
		&&& The ``Fall” term shall be defined from September 1st to December 31st of the Affiliation Year.
		&&& The ``Winter ” term shall be defined as January 1st to April 31st of the Affiliation Year.
		&&& The ``Summer” term shall be defined as May 1st to August 31st of the Affiliation Year.
		&&& The ``Affiliation Committee” is defined in Section 4.12.6 of Bylaw 1.
	&& Clubs may obtain affiliation with the Society in one of the following status:
		&&& Trial Status (Lasting a duration of 4 months with the potential for extension as deemed necessary by the VP Student Life)
		&&& Full Status ( Lasting from time of approval to the end of the current Affiliation year)
	&& Undergraduate student groups who have not been granted affiliation by the Society within two (2) years after the expiration of their previous affiliation year can only be considered for the Trial Status affiliation.
	&& The Society does not endorse the views expressed by those of affiliated clubs.
		&&& No club shall purport to speak for the Society.
\end{easylist}

\section{Standards for Affiliation}
\begin{easylist}
\ListProperties(Start2=0)
	&& Affiliated clubs must seek to make a positive contribution to the academic and/or co-curricular lives of University of Toronto undergraduate Engineering students.
	&& Affiliated clubs must provide benefit to the undergraduate engineering student body by developing members’ skills and/or improving the student experience.
	&& Affiliated clubs must not discriminate on the basis of race, ancestry, place of origin, colour, ethnic origin, citizenship, creed, sex, sexual orientation, gender identity, gender expression, age, marital status, family status, and/or disability.
	&& Affiliated clubs must operate on a not-for-profit basis and cannot pay any club members for their involvement within the club, such that all revenue is used to support the operations of the club.
		&&& Affiliated clubs may provide subsidies to members to reduce financial barriers to involvement, as approved by the Vice-President Finance.
		&&& Affiliated clubs that charge membership fees must demonstrate that fees are used to support operations of the club.
	&& Affiliated clubs must have at least 10 members who are enrolled as undergraduate engineering students at the University of Toronto.
	&& Affiliated clubs must follow relevant Society bylaws and policies.
	&& Affiliated clubs must make the Society’s Policy on Complaints easily accessible to their members.
	&& Members of affiliated clubs must not:
		&&& Violate the University of Toronto Student Code of Conduct Section B-1.
		&&& Commit other forms of harassment (as defined in the University of Toronto Statement on Prohibited Discrimination and Discriminatory Harassment).
		&&& Commit or attempt to commit other acts deemed inappropriate by both the President and Vice-President Student Life.
		&&& Fail to report violations of this standard of behaviour to an appropriate authority (whether to appropriate individuals the Engineering Society, to appropriate individuals within the University of Toronto, or appropriate law enforcement authorities).
	&& If a club leader is made aware that a member of their club has committed a violation outlined in Section 1.8, they should report the incident to the Ombudsperson in accordance with the Society’s Policy on Complaints.
		&&& In the case that a club leader receives a confidential disclosure they are not given permission to share, the club leader should explain the Society’s complaints resources to the complainant.
	&& If after a report to the Ombudsperson, no action is taken by the club to rectify the situation in a timely manner to the satisfaction of the Ombudsperson, the Ombudsperson will notify the VP Student Life and the President that an affiliation violation has occurred and recommend outcomes including but not limited to removal of the club’s affiliation status as outline in Section 7.
\end{easylist}

\section{Engineering Club Trial Status Standards to be Met}
\begin{easylist}
\ListProperties(Start2=0)
	&& The club must be unique from all other pre-existing Engineering Society affiliated clubs.
		&&& The name, mandate, scope, and focus shall not overlap significantly with that of other clubs, as determined by the Affiliations Committee.
		&&& The name of the club must reflect their purpose and not make reference to any Engineering Society Officers, Project Directorships, Affiliated Clubs, or Associated Entities.
		&&& It is the responsibility of the club applying to ensure that there are no similar clubs already affiliated.
		&&& If similar clubs are applying for Trial Status at the same time, they will be given the following options: prove differentiation by revising their application, merge together, or leave their application as is and compete for Trial Status affiliation.
	&& Trial Status shall be held for 4 months before the club may proceed to the Engineering Club Full Status from Trial Application (Section 4).
		&&& If clubs are not granted Full Status within the balance of their affiliation year, they shall re-apply for Trial Status.
		&&& The summer term does not contribute to the Trial status period.
	&& Once granted Trial Status, clubs will be eligible for all the benefits of Full Status clubs, with restrictions on their allocated budget as per the Engineering Society Policy on Finances.
\end{easylist}

\section{Engineering Club Trial Status Application}
\begin{easylist}
\ListProperties(Start2=0)
	&& A club representative shall complete and submit the following to the Affiliations Committee to apply for trial status:
		&&& Club proposal letter detailing:
			&&&& Club name.
			&&&& Club description.
			&&&& Contact name and email address.
			&&&& Type of club (select one: cultural, design/competition team, musical group, fine arts, athletic group, societal/community/charity, hobby/interest, professional development).
			&&&& Mandate of the club.
			&&&& Vision and goals for the club.
			&&&& Difference between this club and existing affiliated clubs.
			&&&& Contribution/benefit to Engineering Society members.
	&& Following the receipt of the application, the Affiliation Committee may request to meet with a club representative.
\end{easylist}

\section{Engineering Club Full Status - from Trial - Application}
\begin{easylist}
\ListProperties(Start2=0)
	&& After 4 months of maintaining trial status a club representative shall complete and submit the following to the Club Affiliation Committee in order to achieve Full Status:
		&&& Description of how the club fulfilled its mandate during the trial period.
		&&& Description of how the club contributed to and benefited the University of Toronto Engineering community.
		&&& A plan to ensure long-term sustainability, including:
			&&&& A draft timeline for the next year for major events and activities.
			&&&& A statement of long-term goals.
			&&&& A plan for leadership turnover (e.g. a plan for elections/appointments, etc).
			&&&& Assessment of potential avenues of partnership or collaboration with other organizations (internal or external to the University of Toronto).
		&&& Detailed description of how the club used the money allocated to it by the Finance Committee, if applicable.
		&&& Constitution or founding document, including:
			&&&& Statement of vision and goals.
			&&&& Executive position structure.
		&&& Member list:
			&&&& The club shall demonstrate that at least ten (10) undergraduate engineering students that function as active members of the club.
			&&&& An active member must have provided their name, UofT email address, program, and graduating year during the previous Affiliation Year prior to the application.
			&&&& Each active member may be contacted to verify they are active members of the club.
	&& Approval of Full Status applications shall be ratified by the Board of Directors.
	&& Full Status will expire at the end of the Affiliation Year and must be renewed.
\end{easylist}

\section{Engineering Club Full Status Standards to be Met}
\begin{easylist}
\ListProperties(Start2=0)
	&& The club may be asked to attend an interview with the Club Affiliation Committee to verify legitimacy, at the discretion of the VP Student Life or President at any time during their affiliation with the Engineering Society.
	&& The club shall not encroach on other clubs’ mandates or scopes.
	&& Full Status will expire at the end of the Affiliation Year and must be renewed.
\end{easylist}

\section{Engineering Club Full Status Annual Renewal Application}
\begin{easylist}
\ListProperties(Start2=0)
	&& In order to renew Full Status a club representative shall complete and submit the following to the Club Affiliation Committee:
		&&& Description and picture for promotional uses by the Society.
		&&& Document with the following information from the past year:
			&&&& Length of time that the club has been active.
			&&&& Length of time that the club has been affiliated with the Society.
			&&&& List of services provided by the Society that the club used.
			&&&& Contributions and benefits to Members.
			&&&& Note of whether the club’s budget will exceed \$500.
		&&& Onus of Differentiation:
			&&&& A detailed explanation describing how this club differs from other similar clubs.
			&&&& If similar clubs are applying for Full Status renewal at the same time they will be given the following options: prove differentiation by revising their application, merge together, or leave their application as is and compete for Full Status renewal.
		&&& Member list:
			&&&& The club shall demonstrate that at least ten (10) undergraduate engineering students that function as active members of the club.
			&&&& An active member must have provided their name, UofT email address, program, and graduating year during the current affiliation year.
			&&&& Each active member may be contacted to verify they are indeed active members of the club.
\end{easylist}

\section{Process of Loss of Engineering Club Status}
\begin{easylist}
\ListProperties(Start2=0)
	&& Full Status Affiliation can only be removed prematurely by a two-thirds majority vote at a meeting of the Board of Directors. Trial Status Affiliation may be removed by the Vice-President Student Life at the next Club Affiliation Committee meeting.
	&& The Vice President Student Life will maintain a record of all violations committed by Affiliated Clubs in the history of their affiliation.
		&&& All violations are either in an ‘active’ or ‘inactive’ state. All violations added to the club’s record shall be in an ‘active’ state. Any violation older than 9 years shall be rendered ‘inactive’.
	&& If the Ombudsperson determines that any affiliated club in either trial status or full status violates the standards set forth in the policy Section 1 (Standards for Affiliation), the following process shall be followed:
		&&& The Vice-President Student Life will alert the club of the details of the violation, ask the club for a second time to rectify the situation in a proper and timely manner and note the violation in their record of affiliation violations.
		&&& If the club does not rectify the situation in a proper and timely manner and the club already has an active violation on their record, the Vice-President Student Life shall submit a motion to the Board of Directors to remove their affiliation status. The club will lose access to funding and resources from the Society until the situation has been rectified.
		&&& If the club decides to appeal the decision of the Vice-President Student Life at the meeting of the Board of Directors, the Board of Directors can decide on one of the following recommended outcomes:
			&&&& Agree with the decision of the Vice-President Student Life and remove the club’s Affiliation Status.
			&&&& Disagree with the decision of the Vice-President Student Life and reinstate the club’s Affiliation Status to what it was before the offense in question. The violation will not be added to the club’s record.
			&&&& Disagree with the decision of the Vice-President Student Life and reinstate the club’s Affiliation Status to what it was before the offense in question. The violation will be added to the club’s record as active.
\end{easylist}

\section{Appendix}
\begin{easylist}
\ListProperties(Start2=0)
&& The following changes were made from the previous version (Revised on August 13, 2015):
	&&& Section 0.2 was added as per December 2015 BoD decision.
	&&& Multiple changes as per November 2017 BoD decision.
	&&& Addition of standard for clubs to adhere to and process for removal of affiliated status.
	&&& Multiple changes as per April 2023 BoD decision.
\end{easylist}

\end{document}
