%Use XeLaTeX to compile.

%%%%%
% SET THIS TO THE DATE THAT THE CURRENT VERSION
% IS PASSED/TO BE PASSED BY THE BOARD OF DIRECTORS
%%%%%
\newcommand{\revdate}{April 12, 2023}

\documentclass[12pt]{article}


%PACKAGES BEGIN
\usepackage{fancyhdr}
\usepackage{geometry}
	\geometry{left=1in, right=1in, top=1in, bottom=1in}
\usepackage{fontspec}
\usepackage[lining,tabular]{ebgaramond}%Special font, use XeLaTeX to compile.
\usepackage{graphicx}
\usepackage{titlesec}
\usepackage[ampersand]{easylist}
\usepackage{hyperref}
% PACKAGES END

% FILE INFO
\author{University of Toronto Engineering Society}
\title{Bylaw 7} %Bylaw number
\date{}
% FILE INFO

% TITLING FORMATTING BEGIN
\titleformat{\section}{\centering\bfseries\large\uppercase}{Chapter\ \thesection \ - }{0ex}{}
\newcommand{\sectionbreak}{\clearpage}
\ListProperties(Numbers=a, Numbers4=l, Numbers5=r, Style2=\bfseries, Start1*=\thesection, Start2=0, FinalMark={.}, Hang=true, Margin2=0cm, Margin3=1cm, Margin4=2.5cm, Margin5=3.5cm, Align=1cm, Align3=1.5cm, Space2=.5cm, Hide4=3, Hide5=4)
\setcounter{section}{-1}
% TITLING FORMATTING END

% HEADER/FOOTER BEGIN
\pagestyle{fancy}
\fancyhf{}
\setlength{\headheight}{42pt}
\lhead{
	\includegraphics{images/logo.png}
	}
\rhead{
	\textbf{University of Toronto Engineering Society} \\
	\textbf{Bylaw 7} \\
	Last Revision: \revdate
	}
\rfoot{
	\thepage
	}
% HEADER/FOOTER END

\begin{document}

% TITLE PAGE BEGIN
\begin{titlepage}
\begin{center}
\topskip0pt
\vspace*{\fill}
\Large\bfseries\uppercase{
	Bylaw 7

	The Discipline Clubs and Class Representatives Bylaw

	University of Toronto Engineering Society
	}
\vspace*{\fill}
\end{center}
\vfill
\begin{flushright}
ADOPTED: March 29, 2016

LAST REVISED: \revdate
\end{flushright}
\end{titlepage}
\pagebreak
% TITLE PAGE END

% TOC BEGIN
\pagenumbering{gobble}
\tableofcontents\let\thefootnote\relax\footnote{{If you have any questions regarding Bylaw 7, please contact the Speaker at speaker@skule.ca.}}
\clearpage
% TOC END

% Setting numbering correctly.
\pagenumbering{arabic}
\setcounter{page}{1}

% Content begin
\section{General}
\vspace{5mm} %Header heigh consistency.
\begin{easylist}
&& General
	&&& All Discipline Clubs shall be privy to the benefits of Affiliated Clubs.
	&&& There shall be one (1) Discipline Club per program of study as defined by the Faculty, with the exception of the Electrical and Computer Engineering Discipline club which shall represent both Electrical Engineering and Computer Engineering.
		&&&& The TrackOne Committee shall serve as the Discipline Club for TrackOne General Engineering.
	&&& The Discipline Club shall be represented by the Club Chair or Co-Chairs.
	&&& Each Discipline Club shall be required to have an Academic Director (or equivalent
role), with whom the Vice-President Academic shall consult on relevant matters of
academic advocacy.
		&&&& Further details regarding this position and its structure can be found in the Constitution of the respective Discipline Club.
\end{easylist}

\section{Finances}
\begin{easylist}
\ListProperties(Start2=0)
&& Finances
	&&& Financial support for Discipline Clubs from the Engineering Society shall be allocated according to the following formula: \$1000 per discipline, plus \$4.50 per student. If the Engineering Society increases membership fees to adjust for cost of living, as per Bylaw 1 Section 1.2.4, then the funding for Discipline Clubs shall be adjusted by the same percentage.
	&&& Financial support shall be delivered in the form of three cheques.
	&&& All discipline club funding shall be withheld until its release is approved by the Board. The Board may decline to release funding for the following reasons:
		&&&& The club has graduate pranks from previous years which are in a state of disrepair.
		&&&& The club does not have a constitution that has been ratified by the Board or is not in compliance with its Board-approved constitution.
		&&&& The club does not have an operating budget that has been approved by the Finance Committee.
		&&&& The club's chair or co-chairs are not in their graduating year.
		&&&& The club has not run elections in compliance with the Society's bylaws.
			&&&&& Each Discipline Club Constitution shall clearly state that the Discipline Club Chair must be elected through the Engineering Society, barring extraneous circumstances, as defined in Bylaw 3.
	&&& In cases where no Chair has been elected, an Interim Chair may be appointed and the requirements for funding in 1.0.3.d and 1.0.3.e be waived.
		&&&& Should a Discipline Club choose to appoint an Interim Chair, the process for doing so must be clearly outlined in their constitution.
		&&&& In cases where an Interim Chair is appointed, the position of Chair must be opened for nominations in the next applicable election cycle regardless.
	&&& A two-thirds majority of members of the Board of Directors, not of the discipline in question, at a meeting of the Board of Directors shall be required to release a club's funding.
	&&& A Discipline Club is required, upon the request of the Board of Directors, to release a current budget and up-to-date record of financial transactions within thirty (30) days of such a request.
\end{easylist}

\section{TrackOne Committee}
\begin{easylist}
\ListProperties(Start2=0)
&& Exceptions to Chapter 1 for the TrackOne Committee
	&&& The per-discipline funding, as described in Section 1.0.1, shall be \$250 instead of \$1000.
	&&& In addition to Section 1.0.3.a, the Board may decline to release funding if the TrackOne Committee has any pranks from previous years which are in a state of disrepair.
	&&& The TrackOne Committee is exempt from the requirements of Section 1.0.3.d.
		&&&& For clarity, the TrackOne Committee Chair need not be in their graduating year.
	&&& The TrackOne Committee is exempt from the requirements of Section 1.0.4.b.
		&&&& For clarity, in cases where an Interim Chair is appointed, the position of TrackOne Committee Chair need not be opened for nominations in the next applicable election cycle.
\end{easylist}

\section{Class Representatives}

\begin{easylist}
\ListProperties(Start2=0)
&& Class Representatives
	&&& There shall be at most two (2) Class Representatives per discipline per year. Class Representatives for a PEY constituency shall be optional at the discretion of the appropriate Discipline Club.
		&&&& Class Representatives shall have the option to be a Faculty Council Representative, holding
one (1) vote on the Council.
			&&&&& If there are two (2) Class Representatives for a particular discipline or year, at most one may hold a vote on Faculty Council as determined by the representatives themselves. This decision shall be communicated to the Vice-President Academic at least three weeks prior to  the first Faculty Council meeting of their term. In the event Class Representatives do not
communicate a decision to the Vice-President Academic within this time frame, then the Vice-President Academic shall appoint a Class Representative to Faculty Council by their own prerogative.
			&&&&& If Class Representatives do not wish to participate in Faculty Council, they must communicate this decision to the Vice-President Academic at least three weeks prior to the first Faculty Council meeting of their term. In the event of a Class Representative choosing not to participate in Faculty Council, the Vice-President Academic shall appoint a Faculty Council Representative through an application process made accessible to all students in the relevant year and discipline.
	&&& A Class Representative shall be automatically recalled upon cessation of their membership
of the constituency that elected them.
	&&& The role of a class representative includes, but is not limited to:
		&&&& Regularly soliciting feedback from their constituency regarding academic matters and present said feedback to the appropriate parties (e.g. Professors, Teaching Assistants, and/or Relevant Faculty Members).
		&&&& Organizing events and activities as required by their Discipline Club.
		&&&& Communicating with their respective Discipline Clubs.
		&&&& Presenting important announcements on behalf of the Vice President Communications to their constituency.
		&&&& Preparing a transition document for onboarding future class representatives of their respective constituency.
	&&& At the discretion of their Discipline Club Chair and the Vice-President Academic or the President, a Class Representative shall be moved to be recalled at a subsequent meeting of the Board under the following circumstances:
		&&&& Cessation of their membership of the constituency or constituencies which elected them.
		&&&& Neglect of their duties as described in Section 3.0.3.
	&&& A Class Representative may be recalled by referendum of the constituency that elected them, which shall be called by a minimum of ten (10) members of that constituency.
		&&&& The names and signatures of those constituents who would like to call the referendum must be submitted in writing to the Chief Returning Officer.
		&&&& In the case that the constituency is smaller than twenty (20) members, the referendum may be called by no less than 50\% of the constituency.
		&&&& If a referendum is called, the Class Representative in question must be notified by email by the Chief Returning Officer at least seven (7) days prior to the start of the referendum.
		&&&& If a referendum is called, the entire constituency that is eligible to vote must be notified by email at the start of the voting period, which shall be at least three (3) days.
		&&&& Any person explicitly advocating for or against recalling the Class Representative in question shall be held to the same standards as referendum delegates as per Bylaw 3, and penalties may be imposed at the discretion of the Chief Returning Officer.

	\end{easylist}

\end{document}
